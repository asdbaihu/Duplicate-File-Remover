A command line utility that {\bfseries recursively} scans the set directory to find exact duplicate files inside all the sub-\/directories. The files so found can be listed in an output file. If required the duplicates can also be removed, thereby preserving a single unique file.

{\bfseries Features\+:}
\begin{DoxyEnumerate}
\item Recursive scan of the set directory.
\item Generate list of duplicate files.
\item Scan all files or filter based on file extension.
\end{DoxyEnumerate}

{\bfseries Future Plans\+:}
\begin{DoxyEnumerate}
\item Set a recursive scan depth for the set directory.
\item A way to exclude certain directories.
\item A way to include only some directories.
\item Robust error handling for synchronization issues.
\item Make the program interactive.
\end{DoxyEnumerate}

\begin{quote}
Caution\+: As of now, there is no way to select which of the duplicate files will be preserved. The selection happens on the order in which they are loaded into {\ttfamily std\+::map}. The first file is the one which is preserved. \end{quote}


\subsection*{Downloading and Building}


\begin{DoxyEnumerate}
\item Clone the project\+:

{\ttfamily git clone \href{https://github.com/vishal-wadhwa/Duplicate-File-Remover.git}{\tt https\+://github.\+com/vishal-\/wadhwa/\+Duplicate-\/\+File-\/\+Remover.\+git}}
\item Change directory to src\+:

{\ttfamily cd src}
\item Build project using Make utility\+:

{\ttfamily make main}
\item Run it (See \href{#use}{\tt Usage})\+:

{\ttfamily ./main ...}
\end{DoxyEnumerate}

\subsection*{Testing}


\begin{DoxyEnumerate}
\item From the root directory of the project go to test directory\+:

{\ttfamily cd test}
\item Build tests using Make utility.

{\ttfamily make test}
\item Run them tests, bruh\+:

{\ttfamily ./test}
\end{DoxyEnumerate}

You should see {\itshape OK} if all the tests pass and then you can go on to using the program. ;)

\subsection*{\label{_use}%
Usage}


\begin{DoxyEnumerate}
\item Use {\ttfamily -\/d} switch to set the directory to be scanned.
\item Use {\ttfamily -\/e} switch to provide a list of extensions to filter the files scanned.
\item Use {\ttfamily -\/l} switch to generate an output(log) file. If this switch is not followed by a name/path, then a default file {\itshape dupl\+\_\+file.\+txt} is generated in the current directory.
\item Use {\ttfamily -\/r} switch to remove the duplicates and keep only one copy.
\item Use {\ttfamily -\/h} switch to display this help\+:
\end{DoxyEnumerate}


\begin{DoxyCode}
1 Usage: ./main -d [DIRECTORY]
2 or: ./main -d [DIRECTORY] -e [EXTENSIONS]...
3 or: ./main -d [DIRECTORY] -l [OUTFILE]
4 
5 Scan the provided directory and its sub-directories recursively and find duplicates.
6 
7 Not using either of -l or -r switch is pointless as no action is performed.
8 
9 -d switch is necessary to set the search directory.
10 
11 Other switches:
12     -d      provided argument is the directory to be scanned.
13     -e      following arguments treated as extensions.
14     -l      generate file list (default file: "dupl\_file.txt").
15     -h      prints this help.
16     -r      remove the duplicates so found.
\end{DoxyCode}
 \begin{quote}
Note\+: Use {\ttfamily sudo} if required. \end{quote}


\subsection*{Examples}


\begin{DoxyEnumerate}
\item {\ttfamily ./main -\/d ./ -\/l -\/r}
\item {\ttfamily ./main -\/e png jpg jpeg -\/d ./../ -\/l log.\+out}
\item {\ttfamily ./main -\/d ./ -\/r} 
\end{DoxyEnumerate}